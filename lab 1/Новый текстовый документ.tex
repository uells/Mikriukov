\documentclass[4apaper,11pt]{article}

\usepackage{amsmath,amssymb,amsthm}
\usepackage{algorithm}
\usepackage[noend]{algpseudocode} 

%---enable russian----

\usepackage[utf8]{inputenc}
\usepackage[russian]{babel}


% PROBABILITY SYMBOLS
\newcommand*\PROB\Pr 
\DeclareMathOperator*{\EXPECT}{\mathbb{E}}


% Sets, Rngs, ets 
\newcommand{\N}{{{\mathbb N}}}
\newcommand{\Z}{{{\mathbb Z}}}
\newcommand{\R}{{{\mathbb R}}}
\newcommand{\Zp}{\ints_p} % Integers modulo p
\newcommand{\Zq}{\ints_q} % Integers modulo q
\newcommand{\Zn}{\ints_N} % Integers modulo N

% Landau 
\newcommand{\bigO}{\mathcal{O}}
\newcommand*{\OLandau}{\bigO}
\newcommand*{\WLandau}{\Omega}
\newcommand*{\xOLandau}{\widetilde{\OLandau}}
\newcommand*{\xWLandau}{\widetilde{\WLandau}}
\newcommand*{\TLandau}{\Theta}
\newcommand*{\xTLandau}{\widetilde{\TLandau}}
\newcommand{\smallo}{o} %technically, an omicron
\newcommand{\softO}{\widetilde{\bigO}}
\newcommand{\wLandau}{\omega}
\newcommand{\negl}{\mathrm{negl}} 

% Misc
\newcommand{\eps}{\varepsilon}
\newcommand{\inprod}[1]{\left\langle #1 \right\rangle}


\newcommand{\handout}[5]{
	\noindent
	\begin{center}
		\framebox{
			\vbox{
				\hbox to 5.78in { {\bf Научно-исследовательская практика} \hfill #2 }
				\vspace{4mm}
				\hbox to 5.78in { {\Large \hfill #5  \hfill} }
				\vspace{2mm}
				\hbox to 5.78in { {\em #3 \hfill #4} }
			}
		}
	\end{center}
	\vspace*{4mm}
}

\newcommand{\lecture}[4]{\handout{#1}{#2}{#3}{Scribe: #4}{Название темы #1}}

\newtheorem*{theorem}{Теорема Вильсона}
\newtheorem{lemma}{Лемма}
\newtheorem{definition}{Определение}
\newtheorem{corollary}{Следствие}
\newtheorem{fact}{Факт}

% 1-inch margins
\topmargin 0pt
\advance \topmargin by -\headheight
\advance \topmargin by -\headsep
\textheight 8.9in
\oddsidemargin 0pt
\evensidemargin \oddsidemargin
\marginparwidth 0.5in
\textwidth 6.5in

\parindent=10mm
\parskip 1.5ex

\begin{document}
\lecture{Освоение Latex}{Лето 2020}{}{Микрюков Данила}
или $2^{341}\equiv 2\mod 341$. После отмены коэфициента 2, мы переходим к \[2^{340}\equiv 1\mod 341,\] таким образом, обратное утверждение к теореме Ферма неверно.

Исторический интерес к числам вида $2^n-2 $ заключается в утверждении, сделанном китайскими математиками более 25 веков назад, о том, что $n$ является простым тогда и только тогда, когда $n\mid2^n-2$ (на самом деле этот критерий надежен для всех целых чисел $n\leq340$). Излишне говорить, что наш прмер, где $341\mid2^{341-2 }$, хотя $341=11\cdot31$, опровергает гипотезу; Это было обнаружено в 1819 году. Ситуация, в которой $n\mid2^n-2$  встречается достаточно часто, чтобы иметь название: состаное число $n$ называют псевдопростым всякий раз, когда $n\mid2^n-2$. Можно показать, что существует бесконечно много псевдопростых чисел, наименьшими из которых являются 341, 561, 645 и 1105.
\section{Задачи}
\begin{enumerate}
\item Убедитесь,что $18^6\equiv1~(mod~7^k)$ for $k=1,2,3$.
\item 
(а) Если $\text{НОД}(a,35)=1$, показать, что $a^{12}\equiv1\mod 35$. [\textsl{Подсказка}: По Теореме Ферма $a^6\equiv1\mod7$ и $a^4\equiv1\mod5$]

(б) Если $\text{НОД}(a,42)=1$, показать, что $168=3\cdot7\cdot8$ делит $a^5\equiv 5\mod5$  

(в) Если $\text{НОД}(a,133)=\text{НОД}(b,133)$, показать что $133\mid a^{18}-b^{18}$
\item Докажите, что существует бесконечно много составных чисел, для которых $a^{n-1}\equiv a\mod n$. [\textsl{Подсказка}: Возьмите $n=2p$, где $p$ нечетное простое число.]
\item Вывести каждую из следующих соответствий 

(а) $a^{21}\equiv a\mod 15$ для любого $a$. [\textsl{Подсказка}: по Теореме Ферма, $a^5\equiv a \mod5$.]

(б) $a^7\equiv a\mod42$ для любого $a$.

(в) $a^{13}\equiv a\mod 3\cdot7\cdot13$ для люого $a$.
\item Для любого целого $a$ показать, что $a^5$ и $a$ имеют одно и то же число измерений
\item Найти число измерений $3^{100}$, пользуясь теоремой Ферма. [\textsl{Подсказска:} Запишите $3^{100}=3(3^9)^{11}$.]
\item Докажите, что при любом положительном целом $n$ выполняются слюдующие соответсвия:

(а) $2^{2n}\equiv 1\mod 3$

(б) $2^{3n}\equiv 1\mod 7$

(в) $2^{4n}\equiv 1\mod 15$
\item 
(а) Пусть $p$ -- простое число, а $\text{НОД}(a,p)=1$. Используйте теорему Ферма, чтобы показать, что $x\equiv a^{p-2}b\mod p$ является решением линейного уравнения $ax\equiv b\mod p$
(б) Учитывая решение из пункта (а), решить линейное уравнение $2x\equiv 1\mod 31$, $6x\equiv 5\mod 11$, $3x\equiv 17\mod 29$.
\item Преположив, что $a$ и $b$ -- целые числа, не делящиеся на простое число $p$, установите следующее:

(а) Если $a^p\equiv b^p\mod p$, то $a\equiv b\mod p$.

(б) Если $a^p\equiv b^p\mod p$, то $a\equiv b\mod p^2$. [\textsl{Подсказка:} Из пункта (а) $a=b+pk$ для некоторого k, так что $a^p-b^p=(b+pk)^p-b^p$; теперь покажем, что $p^2$ делит последнее выражение]
\item Используя Теорему Ферма доказать, что если $p$ -- простое нечетное число, то

(a) $1^{p-1}+2^{p-1}+3^{p-1}+\dots+(p-1)^{p-1}\equiv -1\mod p$

(б) $1^p+2^p+3^p+\dots+(p-1)^p\equiv 0 \mod p$. [\textsl{Подсказка:} Вспомните тождество $1+2+3+\dots+(p-1)=p(p-1)/2$.]
\item Докажите, что если $p$ простое нечетное число и $k$ любое целое число, удовлетворяющее неравенству $1\leq k \leq p-1$, то биноминальный коэфициент \[\begin{pmatrix}
p-1\\k
\end{pmatrix}\equiv (-1)^k\mod p\]
\item Предположим, что $p$ и $q$ различные нечетные простые числа такие, что $p-1\mid q-1$.\\ Если $\text{НОД}(a,pq)=1)$, показать, что $a^{q-1}\equiv 1\mod pq$.
\item Если $p$ и $q$ -- различные простые числа, доказать, что 
\[p^{q-1}+q^{p-1}\equiv 1\mod pq\]
\item Покажите, что целые $1729=7\cdot13\cdot19$ и $1905=3\cdot5\cdot127$ являются псевдопростыми.
\item Покажите, что $561\mid 2^{561}-2$ и $561\mid 3^{561-3}$; это вопрос без ответа,
существует ли бесконечно много составных чисел $n$ со свойством $n\mid2^n-2$ и $n\mid 3^n-3$.
\end{enumerate}

\section{Теорема Вильсона}
Теперь мы обратимся к еще одной вехе в развитии теории чисел. Английский математик Эдвард Уоринг (1741-1793) в своих <<Размышлениях об алгербре>> 1770 года выдвинул несколько новых теорем. Главным из них является интересное свойство простых чисел, сообщенное ему одним из бывших учеников,  Джоном Уилсоном. Свойство заключается в следующем: если $p$ -- простое число, то $p$ делит $(p-1)!+1$. Уилсон догодался об этом,опираясь на численные расчеты, но ни он, ни Уоринг не знали, как это доказать. Признав свою неспособность сделать это, Уоринг добавил: <<теоремы такого рода будет очень трудно доказать без специальных обозначений>>. (Читая отрывок, Гаусс произнесс свой красноречивый комментарий о понятиях против нотаций)	Несмотря на пессимистический прогноз Уоринга, Лагранж вскоре после этого (в 1771 г.) дал доказательтво <<Теореме Вильсона>>, заметив при этом и обратное. Было бы, пожалуй, более справедливо назвать теорему в честь Лейбница, поскольску есть свидетельсва, что он знал о результате почти столетие назад, но ничего не опупликовал по этому вопросу.
\begin{theorem}
Если $p$ -- простое чило,то $(p-1)!\equiv -1\mod p$	
	\begin{proof}
	Отбрасывая случаи $p=2$ и $p=3$, как очевидные, возьмем $p>3$. Предположим, что $a$ -- одно из $p-1$ позитивных целых чисел 
	\[1, 2, 3,\dots, p-1\]
	и рассмотрим $ax\equiv 1\mod p$. $\text{НОД}(a,p)=1$. Согласно Теореме 4-7, это выражение имеет одно единственное решение по модулю $p$, следовательно, существует единственное целое число $a\textquoteright$, такое, что $1\leq a\textquoteright\leq p-1$ и $a a\textquoteright \equiv 1\mod p$.
	
	Поскольку $p$ -- простое, $a\textquoteright=a$ тогда и только тогда, когда $a=1$ или $a=p-1$. Дейсвительно, запись $a^2\equiv 1\mod p$ эквивалентна $(a-1)\cdot(a+1)\equiv 0\mod p$. Поэтому либо $a-1\equiv 0\mod p$,  в этом случае $a=1$, либо $a+1\equiv0\mod p$, в этом случае $a=p-1$.
	
	Если мы опустим числа 1 и $p-1$, то получиться сгруппировать оставшиеся целые числа $2,3\dots,p-2$ в пары $a,a\textquoteright$, где $a\neq a\textquoteright$, так, что $aa\textquoteright\equiv 1\mod p$. Когда эти $(з-3)/2$ конгруэнций перемножаются и коэфиценты переставляются, мы получаем
	\[2\cdot3\dots(p-2)\equiv 1\mod p\] или
	\[(p-2)!\equiv 1\mod p\]
	Теперь умножим на $p-1$, чтобы получить:
	\[(p-1)!\equiv p-1\equiv -1\mod p,\] 
	что и требовалось доказать.
	\end{proof}
\end{theorem}
Конкретный пример должен помочь прояснить доказательство теоремы Вильсона. В частности, возьмем $p=13$. Можно разделить на 
\end{document}