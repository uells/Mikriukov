\documentclass[handout]{beamer}
\usepackage{amsmath,amssymb,amsthm}

\usepackage[utf8]{inputenc}
\usepackage[russian]{babel}

\title{Создание презентации с помощью LATEX}
\subtitle{Компьютерная безопасноть(1 курс)}
\author{Микрюков Данила Александрович}
\institute{Балтийский Федеральный Университет}
\date{\today}

\newtheorem*{mylem}{Лемма Евклида}
\newtheorem*{predl}{Основные свойства сравнений}
\begin{document}

\begin{frame}
\titlepage
\end{frame}

\section{Лемма Евклида}
\begin{frame}
\begin{mylem}
	Пусть $a, b, c$ -- целые числа, $a\mid bc$ и $a$ взаимно просто с $b$. Тогда $a\mid c$
	\begin{proof}
	Существуют целые числа $u, v$, такие, что $au + bv = 1$. Умножая последнее равенство на $c$, получаем $acu + bcv = c$. Число a делит левую часть этого равенства и, следовательно, делит правую часть, т.е.$c$. 
	\end{proof}
\end{mylem}
\end{frame}
\section{Основные свойства сравнений}

\begin{frame}
\begin{predl}
	\begin{enumerate}
		\item $x \equiv x \pmod n$, (рефлексивность)
		\item $x \equiv  y \pmod n \Leftrightarrow  y \equiv  x \pmod n$, (симметричность)
		\item $x \equiv y \pmod n, y \equiv z \pmod n \Rightarrow x \equiv  z \pmod n$, (транзитивность)
		\item $x \equiv  y \pmod n, x\textquoteright \equiv  y\textquoteright \pmod n \Rightarrow x \pm x\textquoteright \equiv  y \pm y\textquoteright \pmod n$,
		\item $x \equiv  y \pmod n, x\textquoteright \equiv  y\textquoteright \pmod n \Rightarrow xx\textquoteright \equiv  yy\textquoteright \pmod n$,
		\item $x \equiv  y \pmod n, d\mid n \Rightarrow x \equiv  y \pmod d$,
		\item $x \equiv  y \pmod n, x \equiv  y \pmod m \Rightarrow x \equiv  y \pmod {nm}$
	\end{enumerate}
	 
\end{predl}	
\end{frame}

\end{document}